
\documentclass[12pt]{amsart}
\usepackage{geometry} % see geometry.pdf on how to lay out the page. There's lots.
\usepackage[latin1]{inputenc}
\geometry{a4paper} % or letter or a5paper or ... etc
% \geometry{landscape} % rotated page geometry

% See the ``Article customise'' template for come common customisations

\title{Computergrafik 1. Bericht zur Aufgabe 3}
\author{Alexander Buyanov (806984)}
\author{Phillip Redlich (791806)}

%%% BEGIN DOCUMENT
\begin{document}

\maketitle

\section{Bericht}

In der dritten Aufgabe war eine Beleuchtung f�r den Raytracer zu implementieren. Hierzu waren die Lighting-Klassen
PointLight, SpotLight und DirectionalLight zu erstellen, sowie die Material-Klassen SingleColorMaterial, LamberMaterial und PhongMaterial, zus�tzlich BlingPhongMaterial.
In den Light-Klassen waren die beiden Methoden, illuminate(), welche durch einen Boolean-Wert festlegt ob ein Punkt beleuchtet wird, und directionFrom(), welche den Strahl zur�ckgibt der auf die Lichtquelle zeigt, zu �berschreiben.
In den Material-Klassen war die Methode colorFor() zu �berschreiben. Diese legt die Farbe f�r den �bergebenen Hit in der �bergebenen World fest.

Da uns diesmal die Bearbeitungsreihenfolge vorgegeben wurde ist an unserer L�sungsstrategie nichts besonderes gewesen.
So haben wir zun�chst das Prinzip des Materials umgesetzt, indem wir das SingleColorMaterial und PointLight implementiert haben.
Die Methoden colorFor() und illuminate() der jeweiligen Klasse waren Trivial und haben direkt einen Wert zur�ckgegeben. In diesem Schritt war lediglich die �nderungen der anderen Klassen wichtig.
Die restlichen Klassen haben wir dann sp�ter nachgetragen.

Auch bei der Implementierung gab es keine Besonderheiten, da wir nur die Methoden korrekt ausf�llen mussten.

Unsere Probleme ergaben sich haupts�chlich durch eigene Fehler.
So waren die Berechnungen in den colorFor()-Methoden f�r uns zun�chst nicht trivial.
Nachdem wir jedoch das LambertMaterial korrekt implementieren konnten waren die sp�teren Material-Klassen besser fertig zustellen.

Dementsprechend war unser Zeitaufwand zu Beginn sehr hoch und danach nicht mehr erw�hnenswert. Ausgeschlossen eine l�ngere Fehlersuche.

Insgesamt empfanden wir die Bearbeitung dieser Aufgabe als sehr interessant.

\end{document}